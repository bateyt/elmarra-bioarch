\documentclass[]{book}
\usepackage{lmodern}
\usepackage{amssymb,amsmath}
\usepackage{ifxetex,ifluatex}
\usepackage{fixltx2e} % provides \textsubscript
\ifnum 0\ifxetex 1\fi\ifluatex 1\fi=0 % if pdftex
  \usepackage[T1]{fontenc}
  \usepackage[utf8]{inputenc}
\else % if luatex or xelatex
  \ifxetex
    \usepackage{mathspec}
  \else
    \usepackage{fontspec}
  \fi
  \defaultfontfeatures{Ligatures=TeX,Scale=MatchLowercase}
\fi
% use upquote if available, for straight quotes in verbatim environments
\IfFileExists{upquote.sty}{\usepackage{upquote}}{}
% use microtype if available
\IfFileExists{microtype.sty}{%
\usepackage{microtype}
\UseMicrotypeSet[protrusion]{basicmath} % disable protrusion for tt fonts
}{}
\usepackage[margin=1in]{geometry}
\usepackage{hyperref}
\hypersetup{unicode=true,
            pdftitle={Bioarchaeology at Umm el-Marra},
            pdfauthor={E. K. Batey, III},
            pdfborder={0 0 0},
            breaklinks=true}
\urlstyle{same}  % don't use monospace font for urls
\usepackage{natbib}
\bibliographystyle{apalike}
\usepackage{longtable,booktabs}
\usepackage{graphicx,grffile}
\makeatletter
\def\maxwidth{\ifdim\Gin@nat@width>\linewidth\linewidth\else\Gin@nat@width\fi}
\def\maxheight{\ifdim\Gin@nat@height>\textheight\textheight\else\Gin@nat@height\fi}
\makeatother
% Scale images if necessary, so that they will not overflow the page
% margins by default, and it is still possible to overwrite the defaults
% using explicit options in \includegraphics[width, height, ...]{}
\setkeys{Gin}{width=\maxwidth,height=\maxheight,keepaspectratio}
\IfFileExists{parskip.sty}{%
\usepackage{parskip}
}{% else
\setlength{\parindent}{0pt}
\setlength{\parskip}{6pt plus 2pt minus 1pt}
}
\setlength{\emergencystretch}{3em}  % prevent overfull lines
\providecommand{\tightlist}{%
  \setlength{\itemsep}{0pt}\setlength{\parskip}{0pt}}
\setcounter{secnumdepth}{5}
% Redefines (sub)paragraphs to behave more like sections
\ifx\paragraph\undefined\else
\let\oldparagraph\paragraph
\renewcommand{\paragraph}[1]{\oldparagraph{#1}\mbox{}}
\fi
\ifx\subparagraph\undefined\else
\let\oldsubparagraph\subparagraph
\renewcommand{\subparagraph}[1]{\oldsubparagraph{#1}\mbox{}}
\fi

%%% Use protect on footnotes to avoid problems with footnotes in titles
\let\rmarkdownfootnote\footnote%
\def\footnote{\protect\rmarkdownfootnote}

%%% Change title format to be more compact
\usepackage{titling}

% Create subtitle command for use in maketitle
\newcommand{\subtitle}[1]{
  \posttitle{
    \begin{center}\large#1\end{center}
    }
}

\setlength{\droptitle}{-2em}

  \title{Bioarchaeology at Umm el-Marra}
    \pretitle{\vspace{\droptitle}\centering\huge}
  \posttitle{\par}
    \author{E. K. Batey, III}
    \preauthor{\centering\large\emph}
  \postauthor{\par}
      \predate{\centering\large\emph}
  \postdate{\par}
    \date{02 December, 2018}

\usepackage{booktabs}
\usepackage{amsthm}
\makeatletter
\def\thm@space@setup{%
  \thm@preskip=8pt plus 2pt minus 4pt
  \thm@postskip=\thm@preskip
}
\makeatother

\begin{document}
\maketitle

{
\setcounter{tocdepth}{1}
\tableofcontents
}
\chapter*{Preface}\label{preface}
\addcontentsline{toc}{chapter}{Preface}

\subsection{}\label{section}

\subsection{}\label{section-1}

All text and code for statistical analyses presented in this report are
hosted at \href{https://www.github.com/bateyt/elmarra-bioarch}{GitHub}.
All data and images are located in Dropbox.

\section*{Acknowledgements}\label{acknowledgements}
\addcontentsline{toc}{section}{Acknowledgements}

I would like to thank Dr.~Glenn Schwartz for inviting me join the 2006
field season of the Umm el-Marra Expedition and analyze the human
remains there.

\chapter{Introduction}\label{intro}

\section{General Background}\label{general-background}

The author joined the Tell Umm el-Marra Expedition as bioarchaeologist
for the 2006 field season. During that time, the author collected
osteological data on human skeletal remains recovered from the Early
Bronze mortuary complex, as well as several other skeletal assemblages
\emph{not} associated with the Early Bronze period and/or recovered from
localities outside of the site's acropolis center. All skeletal remains
analyzed in 2006 are curated on site, except a small collection of
skeletal and dental remains, now curated by Glenn Schwartz, project
co-director.

Barbara Stuart provided initial field descriptions of the human remains,
and other preliminary assessments were made by Bruno Frohlich and Judith
Littleton of the Smithsonian Institution \citep{schwartz2007hidden}. The
goal was to re-analyze all human remains recovered from the tomb complex
in order to investigate a number of topics, including demography, diet,
health and paleopathology, possible familial relationships, and
lifestyle reconstruction.

\section{Methods}\label{methods}

All data were collected by the author according to protocols outlined in
\emph{Standards for Data Collection from Human Skeletal Remains}
\citep{standards1994}. The following report includes a summary of those
observations and some results of recent analyses of the data.

\chapter{Early Bronze Remains}\label{earlybronze}

\section{Preservation and Demography}\label{preservation-and-demography}

Generally, preservation of the sample is fair, although there is
differential preservation between and within tombs. Remains from Tombs 1
and 4 exhibit the best overall preservation. Individuals with poor
preservation were generally incomplete, highly fragmentary, or the bones
had undergone significant taphonomic degradation.

For each tomb, the minimum number of individuals (MNI) was determined by
considering repeated skeletal elements and matching commingled remains
based upon diagnostic indicators for age and sex, as well as general
appearance. In some instances, especially if the remains were highly
fragmentary, a large portion of the remains could not be assigned to any
particular individual (e.g., Tombs 3 and 7). Human remains recovered
from the tomb complex (up to the 2006 field season) represent an MNI of
35 individuals (see Table 1).

\section{Paleopathology}\label{paleopathology}

\subsection{General Observations}\label{general-observations}

Due to diff erential preservation, paleopathological data are not
reported for nearly one-third of the sample. Also, because of time
constraints, paleopathology was not recorded for the individuals from
Tomb 8. Of those observed, 13/21 individuals exhibit some type of
pathological lesion.

The most common pathology is periostitis, which is commonly found in
archaeological cemetery samples \citep{ortner2003identification}.
Periostitis occurs in 6/21 of the observable sample and more frequently
on the lower limbs, although the upper limbs are also aff ected. Both
active and healed lesions were observed. Porotic hyperostosis, another
lesion commonly reported in archaeological skeletal samples, is uncommon
in the Umm el-Marra sample. Porotic hyperostosis is often considered a
result of anemia, although its etiology is likely more complicated
\citep{walker2009causes}. No cases of porotic hyperostosis on the
cranial vault occur; however, cribra orbitalia---porosities on the
orbital roof---occur in 2/21 individuals.

\subsection{Selected Cases}\label{selected-cases}

\section{Dietary Reconstruction}\label{dietary-reconstruction}

\subsection{Stable Isotope Analysis}\label{stable-isotope-analysis}

\subsection{Dental Wear}\label{dental-wear}

\section{Dental Nonmetrics}\label{dental-nonmetrics}

Dental nonmectric data were collected using the UADAS recording system
\citep{scott2000anthteeth}. Population distance was measured using
Smith's Mean Measure of Divergence, or MMD \citep{smith1977note}. The
MMD has become a well established method for anthropologists using
nonmetric traints to investigate biological distance between populations
\citep{harris2004calculation}. \citet{soltysiak2011mmd} provides an R
script for calculating the MMD.

\begin{table}

\caption{\label{tab:unnamed-chunk-4}MMD Matrix for Umm el-Marra and Comparative Samples}
\centering
\begin{tabular}[t]{l|r|r|r|r|r|r|r|r|r}
\hline
  & SOU & LL & KM & AI & EN & AS & LER & JT & UEM\\
\hline
SOU & 0.00000 & 0.17423 & 0.07711 & 0.02062 & 0.09878 & 0.51591 & 0.45097 & 0.24715 & 1.67747\\
\hline
LL & 0.17423 & 0.00000 & 0.10572 & 0.06543 & 0.02985 & 0.16781 & 0.09523 & 0.10818 & 1.28277\\
\hline
KM & 0.07711 & 0.10572 & 0.00000 & -0.02865 & 0.01760 & 0.19850 & 0.24307 & 0.32065 & 1.79021\\
\hline
AI & 0.02062 & 0.06543 & -0.02865 & 0.00000 & -0.07538 & 0.08728 & 0.19567 & 0.07651 & 1.80810\\
\hline
EN & 0.09878 & 0.02985 & 0.01760 & -0.07538 & 0.00000 & 0.07474 & 0.08974 & 0.15814 & 1.74441\\
\hline
AS & 0.51591 & 0.16781 & 0.19850 & 0.08728 & 0.07474 & 0.00000 & 0.02527 & 0.18342 & 1.46623\\
\hline
LER & 0.45097 & 0.09523 & 0.24307 & 0.19567 & 0.08974 & 0.02527 & 0.00000 & 0.12811 & 1.52949\\
\hline
JT & 0.24715 & 0.10818 & 0.32065 & 0.07651 & 0.15814 & 0.18342 & 0.12811 & 0.00000 & 1.25275\\
\hline
UEM & 1.67747 & 1.28277 & 1.79021 & 1.80810 & 1.74441 & 1.46623 & 1.52949 & 1.25275 & 0.00000\\
\hline
\end{tabular}
\end{table}

Also of interest is the significance matrix.

\begin{table}

\caption{\label{tab:unnamed-chunk-5}Significance Matrix for Umm el-Marra and Comparative Samples}
\centering
\begin{tabular}[t]{l|r|r|r|r|r|r|r|r|r}
\hline
  & SOU & LL & KM & AI & EN & AS & LER & JT & UEM\\
\hline
SOU & 1.00000 & 0.00523 & 0.16586 & 0.88139 & 0.12800 & 0.00000 & 0.00000 & 0.00021 & 0\\
\hline
LL & 0.00523 & 1.00000 & 0.00372 & 0.61627 & 0.54867 & 0.05027 & 0.00062 & 0.02086 & 0\\
\hline
KM & 0.16586 & 0.00372 & 1.00000 & 1.00000 & 0.68468 & 0.01361 & 0.00000 & 0.00000 & 0\\
\hline
AI & 0.88139 & 0.61627 & 1.00000 & 1.00000 & 1.00000 & 0.56539 & 0.10508 & 0.55820 & 0\\
\hline
EN & 0.12800 & 0.54867 & 0.68468 & 1.00000 & 1.00000 & 0.40909 & 0.01097 & 0.00223 & 0\\
\hline
AS & 0.00000 & 0.05027 & 0.01361 & 0.56539 & 0.40909 & 1.00000 & 0.73422 & 0.04283 & 0\\
\hline
LER & 0.00000 & 0.00062 & 0.00000 & 0.10508 & 0.01097 & 0.73422 & 1.00000 & 0.00005 & 0\\
\hline
JT & 0.00021 & 0.02086 & 0.00000 & 0.55820 & 0.00223 & 0.04283 & 0.00005 & 1.00000 & 0\\
\hline
UEM & 0.00000 & 0.00000 & 0.00000 & 0.00000 & 0.00000 & 0.00000 & 0.00000 & 0.00000 & 1\\
\hline
\end{tabular}
\end{table}

Also of interest is the standard deviation matrix.

\begin{table}

\caption{\label{tab:unnamed-chunk-6}Standard Deviation Matrix for Umm el-Marra and Comparative Samples}
\centering
\begin{tabular}[t]{l|r|r|r|r|r|r|r|r|r}
\hline
  & SOU & LL & KM & AI & EN & AS & LER & JT & UEM\\
\hline
SOU & 0.08426 & 0.06239 & 0.05565 & 0.13820 & 0.06490 & 0.10336 & 0.04772 & 0.06666 & 0.06900\\
\hline
LL & 0.06239 & 0.04422 & 0.03644 & 0.13056 & 0.04977 & 0.08572 & 0.02782 & 0.04682 & 0.04936\\
\hline
KM & 0.05565 & 0.03644 & 0.02896 & 0.12668 & 0.04334 & 0.08045 & 0.02028 & 0.03980 & 0.04228\\
\hline
AI & 0.13820 & 0.13056 & 0.12668 & 0.23607 & 0.14516 & 0.15183 & 0.12073 & 0.13067 & 0.13382\\
\hline
EN & 0.06490 & 0.04977 & 0.04334 & 0.14516 & 0.06158 & 0.09054 & 0.03528 & 0.05171 & 0.05538\\
\hline
AS & 0.10336 & 0.08572 & 0.08045 & 0.15183 & 0.09054 & 0.13952 & 0.07443 & 0.09056 & 0.09436\\
\hline
LER & 0.04772 & 0.02782 & 0.02028 & 0.12073 & 0.03528 & 0.07443 & 0.01184 & 0.03160 & 0.03415\\
\hline
JT & 0.06666 & 0.04682 & 0.03980 & 0.13067 & 0.05171 & 0.09056 & 0.03160 & 0.05160 & 0.05341\\
\hline
UEM & 0.06900 & 0.04936 & 0.04228 & 0.13382 & 0.05538 & 0.09436 & 0.03415 & 0.05341 & 0.05664\\
\hline
\end{tabular}
\end{table}

\chapter{Non-Early Bronze Remains}\label{non-early-bronze-remains}

During the 2006 excavations, some remains from non-Early Bronze contexts
were recovered and some field analysis completed.

\chapter{Summary and Conclusion}\label{summary}

In 2006, the author analyzed human skeletal remains from an Early Bronze
Age mortuary complex at Tell Umm el-Marra, Syria.

\bibliography{book.bib,packages.bib}


\end{document}
